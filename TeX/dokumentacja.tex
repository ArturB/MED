\documentclass{article}

%% Language and font encodings
\usepackage[utf8x]{inputenc}
\usepackage[T1]{fontenc}
\usepackage{polski}
\usepackage{hyperref}
\usepackage{listings}

%% Useful packages
\usepackage{amsmath}
\usepackage{graphicx}
\usepackage[colorinlistoftodos]{todonotes}

\title{Dokumentacja MED 17Z - wybór atrybutów}
\author{Artur M. Brodzki, Mateusz Orzoł}

\begin{document}
\maketitle

\section{Wstęp} \label{sec:intro}
Zadaniem jest znalezienie na zbiorach danych atrybutów mających największy wpływ na wynik zadania klasyfikacji. Program ma działać w trybie wsadowym, z linii komend i mieć możliwość załadowania każdego z trzech zbiorów danych z repozytorium UCI:
\begin{itemize}
	\item Adults: \url{:http://archive.ics.uci.edu/ml/datasets/Adult}
	\item Flags: \url{ http://archive.ics.uci.edu/ml/datasets/Flags}
	\item PrimaryTumor: \url{ http://archive.ics.uci.edu/ml/datasets/Primary+Tumor}
\end{itemize}

Dla zadanego zbioru danych, program ma wypisać wagę każdego atrybutu w zbiorze - im większa waga, tym lepiej używać danego atrybutu do klasyfikacji. Program ma mieć opcję uruchamiania pozwalającą wybrać jeden z dwóch algorytmów działania:
\begin{itemize}
	\item Algorytm prof. Koronackiego
	\item Algorytm "Boruta"
\end{itemize}
Algorytmy te opisuję w sekcji \ref{sec:algs}

\section{Opis wykorzystywanych algorytmów} \label{sec:algs}
Wybór optymalnego atrybutu w ogólności realizuje się poprzez zbudowanie lasu losowego zawierającego wiele drzew. Cały zbiór danych dzieli się na podzbiory i tworzy osobne drzewo dla każdego podzbioru. Każde drzewo z osobna tworzone jest algorytmem SPRINT, opisanym szczegółowo na notatkach z wykładu o klasyfikacji, w wersji z wykorzystaniem współczynnika Giniego. Z kolei sposób wyznaczania najlepszych atrybutów realizowany jest inaczej w zależności od algorytmu.

\subsection{Oznaczenia} \label{sec:symbols}

Oznaczmy zbiór rekordów pewnego zbioru danych przez $D$, a zbiór atrybutów tego zbioru przez $A$. Niech całkowita liczba rekordów w zbiorze danych wynosi $\left|D\right|$, a liczba atrybutów w tym zbiorze wynosi $\left|A\right|$. 

\subsection{Tworzenie drzewa decyzyjnego losowego algorytmem SPRINT}
\begin{enumerate}
	\item Inicjujemy algorytm ze zbiorem danych $D$ oraz progiem na współczynnik Giniego $gthr$. 
	\item Losujemy $n = \lfloor\sqrt{\left|A\right|}\rfloor$ atrybutów zbioru spośród $A$. Wylosowane atrybuty tworzą podzbiór $A_s$. 
	\item Dla każdego $a \in A_s$:
	\begin{enumerate}
		\item Jeżeli $a$ jest atrybutem typu dyskretnego:
		\begin{enumerate}
			\item Tworzymy listę $vals(a)$ wszystkich wartości atrybutu $a$. 
			\item Dla każdej kombinacji bez powtórzeń $c = \{ a_i, a_j, \ldots, a_m \}$ elementów zbioru $vals(a)$ tworzymy podzbiór $D_c = \{ x \in D: x(a) \in c \}$ rekordów o wartości atrybutu $a$ należącej do kombinacji $c$ oraz podzbiór $D_n = \{ x \in D: x(a) \notin c \}$ rekordów o wartości atrybutu $a$ nienależącej do kombinacji $c$. 
		\end{enumerate}
		\item Jeżeli $a$ jest atrybutem typu numerycznego:
		\begin{enumerate}
			\item Sortujemy zbiór $D$ według wartości atrybutu $a$ w porządku niemalejącym. 
			\item Kolejne wartości $v$ atrybutu $a$ przyjmujemy jako progi odcięcia i tworzymy podział zbioru $D$ na dwa podzbiory: $D_c = \{ x \in D: x(a) \le v \}$ oraz $D_n = \{ x \in D: x(a) > v \}$. 
		\end{enumerate}
		\item Dla każdego podziału $\{D_c, D_n\}$ wyznaczamy wartość współczynnika Giniego $gini(D_c, D_n)$ zbioru rekordów po podziale. 
		\item Wybieramy podział $p_{max}(a)$, który minimalizuje współczynnik Giniego na atrybucie $a$. 
	\end{enumerate}
	\item Spośród najlepszych podziałów dla wszystkich atrybutów \\
	$\{ p_{max}(a_1), p_{max}(a_2), \ldots, p_{max}(a_n) \}$ 
	wybieramy podział globalnie najlepszy $p_{max} = \{D_c, D_n\}$ i zapamiętujemy go w tworzonym węźle drzewa. Zapamiętujemy również warunek podziału. 
	\item Lewe i prawe poddrzewo tworzymy rekurencyjnie na zbiorach odpowiednio $D_c$ i $D_n$. 
	\item Rekursja kończy się, gdy współczynnik Giniego na zbiorze danych w węźle jest mniejszy od progu $gthr$ lub gdy w zbiorze pozostał tylko jeden element. Liść jest również tworzony, gdy wszystkie atrybuty decyzyjne rekordów w zbiorze mają identyczne wartości, tzn. zbiór nie daje się podzielić na podzbiory - nawet, jeśli rekordy należą do różnych klas. Jako klasa zawarta w liściu przyjmowana jest klasa najliczniej występująca w zbiorze danych przypadających na liść. W przypadku, gdy kilka klas jest równolicznych, przyjmowana jest klasa występująca w zbiorze danych jako pierwsza. 
\end{enumerate}

\subsection{Algorytm prof. Koronackiego} \label{sec:koronacki}
\begin{enumerate}
	\item Podział na podzbiory: decydujemy się na $\left|A\right|$ drzew w lesie. Do każdego	drzewa przydzielamy $\left|T\right| / \left|A\right|$ rekordów wybieranych drogą losowania ze zwracaniem (czyli rekordy dla jednego drzewa mogą się powtarzać). Ponieważ rekordy się powtarzają, to niektóre nie zostaną wybrane do żadnego drzewa,
	za to użyjemy ich do testowania skuteczności nauczonych drzew. 
	\item Każde drzewo uczymy z wykorzystaniem przydzielonych mu rekordów oraz
	testujemy za pomocą (tego samego dla każdego drzewa) zbioru rekordów
	testujących. Zapamiętujemy dokładność (ang. \emph{accuracy}) klasyfikacji drzewa
	$i$ : $\varphi(i)$. 
	\item Na każdym z $\left|A\right|$ drzew wybieramy jeden z $A$ atrybutów. Wartości tego atrybutu na zbiorze rekordów uczących są losowo permutowane i na tak zmienionym zbiorze ponownie uczymy drzewo i wykonujemy testowanie.	Zapamiętujemy nową (niższą, bo zbiór uczący został zmieniony) dokładność klasyfikacji $\varphi'(i)$. 
	\item Różnica $w(i) = \varphi(i) − \varphi'(i)$ oznacza wagę danego atrybutu.
\end{enumerate}

\subsection{Algorytm \emph{Boruta}}
\begin{enumerate}
	\item Dokonujemy podziału na podzbiory identycznie jak w algorytmie prof. Koronackiego. 
	\item Replikacja: wszystkie kolumny-atrybuty w danych sa replikowane. W	nowym zbiorze danych mamy więc $2\left|A\right|$ atrybutów: każdy atrybut $a_i$ ma swoją kopię $a'_i$. 
	\item Kopie atrybutów mają losowo permutowane wartości. 
	\item Następnie obliczamy wagę każdego atrybutu $w(i) = \varphi(i) − \varphi'(i)$ (replikowanych i niereplikowanych) tak samo jak w algorytmie Koronackiego. 
	\item  Jako wagę atrybutu $i$ przyjmujemy różnicę pomiędzy wagą Koronackiego jego zwykłej wersji $w(i)$ a wagą jego zreplikowanej wersji $w'(i)$.
\end{enumerate}

\section{Podział zadań pomiędzy członków zespołu}

Artur Brodzki:
\begin{enumerate}
	\item Implementacja algorytmu SPRINT w wersji dostosowanej do tworzenia lasów losowych
	\item Testowanie oraz pisanie dokumentacji końcowej projektu
\end{enumerate}
Mateusz Orzoł:
\begin{enumerate}
	\item Implementacja algorytmów prof. Koronackiego oraz algorytmu \emph{Boruta}
	\item Implementacja modułu parsowania plików z danymi z formacie CSV
	\item Interfejs CLI programu - parsowanie parametrów linii komend
\end{enumerate}

\section{Obsługa programu}

Program napisany został jako projekt środowiska Visual Studio 2015. Kod źródłowy programu znajduje się na repozytorium GitHuba pod adresem \url{https://github.com/ArturB/MED} Do jego skompilowania wymagany jest pakiet Windows SDK w wersji 10.0.14393.0 oraz kompilator MSVC 2015 w wersji 140. Prekompilowany plik wykonywalny MED.exe na architekturę x86-64 znajduje się w podkatalogu x64/Release. Dostępny jest równiez intalator programu w formie archiwum ZIP, zawierający plik wykonywalny oraz przykładowe zbiory danych: można go pobrać z repozytorium GitHuba w zakładce Releases: \url{https://github.com/ArturB/MED/releases/tag/1.0.3}. Znajdują się tam również pliki .bat umożliwiające uruchomienie programu dla róznych parametrów oraz przykładowe pliki .txt z wynikami. 

Program uruchamia się, podając kilka parametrów:

\begin{lstlisting}
	MED.exe <datafile> <decision-attr> <algorithm>
\end{lstlisting}

\begin{itemize}
	\item \emph{datafile}: nazwa pliku z danymi. W podkatalogu x64/Release znajdują się już trzy pliki z danymi: 
	\begin{itemize}
		\item adult.txt
		\item flag.txt
		\item tumor.txt
	\end{itemize}
	Umożliwia to bezpośrednie testowanie programu na trzech wymienionych w sekcji \ref{sec:intro} zbiorach danych. 
	\item \emph{decision-attr}: indeks atrybutu, który określa klasę rekordu. Atrybuty numerowane są od 0. Atrybut określający klasę rekordu musi być atrybutem typu dyskretnego. 
	\item \emph{algorithm}: algorytm tworzenia lasu losowego. Dla $\emph{algorithm} = 1$ wykonywany jest algorytm prof. Koronackiego, a dla $\emph{algorithm} = 2$ wykonywany jest algorytm \emph{Boruta}. 
\end{itemize}

W przypadku uruchomienia programu bez parametrów, przyjmowane są domyślne wartości:

\begin{lstlisting}
	MED.exe adult.txt 1 1
\end{lstlisting}

Zatem klasyfikacja wykonywana jest na zbiorze Adult-UCI, z atrybutem \emph{workclass} jako klasą rekordu, metodą prof. Koronackiego. 

Wyniki działania programu wypisywane są na standardowe wyjście, można je przekierować do pliku operatorem >. 

\section{Testy}

\subsection{Wydajność}

Czas działania programu zależy oczywiście od rozmiaru analizowanych danych oraz od wyboru algorytmu tworzenia lasu losowego. Przykładowe czasy wykonania programu na zbiorze Adults-UCI, na komputerze klasy PC z procesorem Intel i5-3570 3,4GHz znajdują się w tabeli \ref{tab:times}. 

\begin{table} 
	\centering
	\begin{tabular}{|c|c|} \hline
		Koronacki & Boruta \\ 
		25.995 s & 37.659 s \\ 
		25.505 & 34.928 \\ 
		23.171 & 42.025 \\ 
		22.417 & 33.551 \\ 
		17.539 & 39.937 \\ 
		21.763 s & 37.819 s \\ 
		24.898 & 33.199 \\ 
		21.67 & 29.482 \\ 
		19.06 & 34.335 \\ 
		28.045 & 41.076 \\ \hline
	\end{tabular}
\caption{Przykładowe czasy wykonania programu na zbiorze danych Adults.}
\label{tab:times}
\end{table}

Przeciętny czas wykonania programu wynosi metodą prof. Koronackiego wynosi 23.11 sek, z odchyleniem standardowym 3.15 sek, natomiast metodą \emph{Boruta} wynosi 36.4 sek, z odchyleniem 3.77 sek. Na uwagę zasługuje fakt, że algorytm \emph{Boruta} wykonuje się ok. 1,57 razy dłużej od algorytmu prof. Koronackiego - wynika to niewątpliwie z faktu, że algorytm \emph{Boruta} wymaga skopiowania wszystkich atrybutów na zbiorze danych - poza tym, algorytmy są do siebie bardzo podobne. Co prawda liczba atrybutów z zbiorze danych jest podwajana, ale w algorytmie lasu losowego nie są analizowane wszystkie atrybutu tylko ich część równa $\lfloor\sqrt{\left|A\right|}\rfloor$. Dla 15 atrybutów wybieramy więc zawsze 3, a dla 30 - 5. Stosunek tych liczb 1.6 odpowiada niemal dokładnie róznicy wykonania algorytmu Koronackiego i \emph{Boruta} (1.57). Niewielka róznica wynika zapewne z faktu, że oprócz samego tworzenia lasu losowego, w programie występuje pewien czynnik stały, związany z parsowaniem pliku CSV z danymi. 

\subsection{Wyniki}

Skuteczność algorytmu SPRINT wypada zadowalająco. Przeprowadzono następujące testy budowanych klasyfikatorów:
\begin{itemize}
	\item Przewidywanie typu pracodawcy na zbiorze Adult: przeciętny współczynnik skuteczności (ang. \emph{accuracy}) dla stworzonego drzewa wynosił pomiędzy 0.6 a 0.8, choć rzadko pojawiały się też obserwacje odstające ze skutecznością 0.3 - 0.4. Skuteczność klasyfiatora losowego wynosiłaby 0.125. 
	\item Przewidywanie kontynentu państwa na podstawie koloru flagi: skuteczność średnio wyniosła między 0.15 a 0.3, czyli tylko nieco więcej niż losowa 0.17. Wynika to jednak z niewielkiej liczności zbiorów uczączych dla drzew: tworząc 30-elementowy las losowy i mając 194 rekordy w zbiorze danych, na jedno drzewo przypada jedynie ok. 5-6 przykładów uczących. 
	\item Przewidywanie typu nowotworu na podstawie jego umiejscowienia: skuteczność średnio wyniosła pomiędzy 0.3 a 0.6, przy 0.05 losowej (są aż 22 typy nowotworów). 
\end{itemize}

Wyniki w każdym przypadku są dość mocno zmienne, co wynika z losowości wprowadzonej do algorytmu SPRINT na potrzeby generowania lasów losowych - randomizowanego doboru analizowanych w każdym węźle atrybutów. 

Mniej zadowalająco wypadły testy samych algorytmów prof. Koronackiego i \emph{Boruta}. Niezależnie od zbioru danych oraz niezależnie od atrybutu, uzyskiwane wagi atrybutów są niemal zawsze równe 0. Czasami pojawiają się niewielkie wartości dodatnie - zazwyczaj dla atrybutu determinującego klasę rekordu w danym zbiorze. Biorąc pod uwagę uzyskane wyniki, nie daje się wyciągnąć konkluzywanych wniosków na temat róznic w rezultatach generowanych przez algorytmy Koronackiego i \emph{Boruta}. 


\end{document}
